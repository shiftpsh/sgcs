% Problem 2
\subsubsection{2.} 다음 상황에 맞는 표본공간을 구하라.
\begin{itemize}
	\item[(1)] ``1"의 눈이 나올 때까지 공정한 주사위를 반복하여 던진 횟수
	\item[(2)] 영하 5도에서 영상 7.5도까지 24시간 동안 연속적으로 기록된 온도계의 눈금의 위치
	\item[(3)] 형광등을 교체한 후로부터 형광등이 나갈 때까지 걸리는 시간
\end{itemize}

\paragraph{Solution.}
\begin{itemize}
	\item[(1)] $S = \left\{ x \middle| x \geq 1 \right\}$
	\item[(2)] $S = \left\{ x \middle| -5 \leq x \leq 7.5 \right\}$
	\item[(3)] $S = \left\{ x \middle| x \geq 0 \right\}$
\end{itemize}

% Problem 5
\subsubsection{5.} 주머니 안에 빨간색과 파란색의 공깃돌이 두 개씩 들어 있는 주머니에서 공깃돌 두 개를 차례로 꺼낸다.
\begin{itemize}
	\item[(1)] 나올 수 있는 공깃돌의 색에 대한 표본공간을 구하라.
	\item[(2)] 공깃돌 두 개가 서로 다른 사건을 구하라.
	\item[(3)] 파란색이 많아야 한 개인 사건을 구하라.
	\item[(4)] 첫 번째 공깃돌이 빨간색이고, 두 번째 공깃돌이 파란색인 사건을 구하라.
\end{itemize}

\paragraph{Solution.} 빨간색 공깃돌이 나오는 사건을 $R$, 파란색 공깃돌이 나오는 사건을 $B$라 하자.
\begin{itemize}
	\item[(1)] $S = \left\{ RR, RB, BR, BB \right\}$
	\item[(2)] $S = \left\{ RB, BR \right\}$
	\item[(3)] $S = \left\{ RR, RB, BR \right\}$
	\item[(4)] $S = \left\{ RB \right\}$
\end{itemize}

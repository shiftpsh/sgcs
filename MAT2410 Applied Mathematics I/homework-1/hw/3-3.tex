% Problem 2
\subsubsection{2.} 확률변수 $X$와 $Y$가 다음 결합확률질량함수를 갖는다고 한다.

\[f\left(x, y\right) = \left\{
\begin{array}{ll}
	\dfrac{1}{3} & \qquad \left(x, y\right) = \left(0, 1\right), \left(1, 0\right), \left(2, 1\right) \\
	0 & \qquad\textrm{다른 곳에서}
\end{array}
\right. \]

\begin{itemize}
  \item [(1)] $X$와 $Y$의 독립성을 조사하라.
  \item [(2)] $X$와 $Y$의 공분산을 구하라.
\end{itemize}

\paragraph{Solution.} \begin{itemize}
  \item [(1)] $X$의 주변확률함수는
\[f_X\left(x\right) = \left\{
\begin{array}{ll}
	\dfrac{1}{3} & \qquad x = 0, 1, 2 \\
	0 & \qquad\textrm{otherwise}
\end{array}
\right. \]

 $Y$의 주변확률함수는
 \[f_Y\left(y\right) = \left\{
\begin{array}{ll}
	\dfrac{1}{3} & \qquad y = 0 \\
	\dfrac{2}{3} & \qquad y = 1 \\
	0 & \qquad\textrm{otherwise}
\end{array}
\right. \]인데, $f\left(0, 0\right) \neq f_X\left(0\right)f_Y\left(0\right)$이므로 $X$와 $Y$는 독립이 아니다.\\
  \item [(2)] $E\left(X\right)=0\times \dfrac{1}{3}+1\times \dfrac{1}{3}+2\times \dfrac{1}{3} = 1$이고 $E\left(Y\right)=0\times \dfrac{1}{3}+1\times \dfrac{2}{3}=\dfrac{2}{3}$이다. 또한 $E\left(XY\right)=0\times 1\times \dfrac{1}{3}+1\times 0\times \dfrac{1}{3}+2\times 1\times \dfrac{1}{3}=\dfrac{2}{3}$이다. $X$와 $Y$의 공분산 $\mathrm{Cov}\left(X, Y\right)=E\left(XY\right)-E\left(X\right)E\left(Y\right)$으로 주어지므로, 공분산은 $0$이다.
\end{itemize}

% Problem 4
\subsubsection{4.} 연속확률변수 $X$와 $Y$의 결합밀도함수가 다음과 같다. \[f\left(x, y\right)=\dfrac{3}{16}\qquad x^2\leq y\leq 4, 0\leq x\leq 2\]

\begin{itemize}
  \item [(1)] $X$와 $Y$의 평균과 표준편차를 구하라.
  \item [(2)] $X$와 $Y$의 공분산을 구하라.
  \item [(3)] $X$와 $Y$의 상관계수를 구하라.
\end{itemize}

\paragraph{Solution.} $x^2\leq y\leq 4 \Rightarrow x\leq \sqrt{y}$

\begin{itemize}
  \item [(1)] $X$의 평균:
\begin{align*}
	E\left(X\right) &= \int_0^2 \int_{x^2}^4 x f\left(x, y\right) \mathop{dy} \mathop{dx} \\
	&= \dfrac{3}{16} \int_0^2 \int_{x^2}^4 x \mathop{dy} \mathop{dx} \\
	&= \dfrac{3}{16} \int_0^2 x\left(4 - x^2\right) \mathop{dx} \\
	&= \dfrac{3}{16} \left[2x^2 - \dfrac{1}{4}x^4\right]_0^2 \\
	&= \dfrac{3}{4}
\end{align*}

$X^2$의 평균:
\begin{align*}
	E\left(X\right) &= \int_0^2 \int_{x^2}^4 x^2 f\left(x, y\right) \mathop{dy} \mathop{dx} \\
	&= \dfrac{3}{16} \int_0^2 \int_{x^2}^4 x^2 \mathop{dy} \mathop{dx} \\
	&= \dfrac{3}{16} \int_0^2 x^2\left(4 - x^2\right) \mathop{dx} \\
	&= \dfrac{3}{16} \left[\dfrac{4}{3}x^3 - \dfrac{1}{5}x^5\right]_0^2 \\
	&= \dfrac{4}{5}
\end{align*}

$Y$의 평균:
\begin{align*}
	E\left(Y\right) &= \int_0^2 \int_{x^2}^4 y f\left(x, y\right) \mathop{dy} \mathop{dx} \\
	&= \dfrac{3}{16} \int_0^2 \int_{x^2}^4 y \mathop{dy} \mathop{dx} \\
	&= \dfrac{3}{16} \int_0^2 8 - \dfrac{1}{2}x^4 \mathop{dx} \\
	&= \dfrac{3}{16} \left[8x - \dfrac{1}{10}x^5\right]_0^2 \\
	&= \dfrac{12}{5}
\end{align*}

$Y^2$의 평균:
\begin{align*}
	E\left(Y\right) &= \int_0^2 \int_{x^2}^4 y^2 f\left(x, y\right) \mathop{dy} \mathop{dx} \\
	&= \dfrac{3}{16} \int_0^2 \int_{x^2}^4 y^2 \mathop{dy} \mathop{dx} \\
	&= \dfrac{1}{16} \int_0^2 64 - x^6 \mathop{dx} \\
	&= \dfrac{1}{16} \left[64x - \dfrac{1}{7}x^7\right]_0^2 \\
	&= \dfrac{48}{7}
\end{align*}

따라서 $X$와 $Y$의 평균과 표준편차는 다음과 같다.

\begin{center}
	\begin{tabular}{r|cc}
			 & $X$ & $Y$ \\
			 \hline
			$\mu$ & $\dfrac{4}{3}$ & $\dfrac{12}{5}$ \\
			\hline
			$\sigma^2$ & $\dfrac{19}{80}$ & $\dfrac{192}{175}$ \\
			\hline
			$\sigma$ & $\dfrac{\sqrt{95}}{20}$ & $\dfrac{8\sqrt{21}}{35}$ \\
			\hline
	\end{tabular}
\end{center}

  \item [(2)] $XY$의 평균:
\begin{align*}
	E\left(XY\right) &= \int_0^2 \int_{x^2}^4 xy f\left(x, y\right) \mathop{dy} \mathop{dx} \\
	&= \dfrac{3}{16} \int_0^2 \int_{x^2}^4 xy \mathop{dy} \mathop{dx} \\
	&= \dfrac{3}{16} \int_0^2 x\left(8 - \dfrac{1}{2}x^4\right) \mathop{dx} \\
	&= \dfrac{3}{16} \left[4x^2 - \dfrac{1}{12}x^6\right]_0^2 \\
	&= 2
\end{align*}

따라서 $\mathrm{Cov}\left(X, Y\right) = E\left(XY\right) - E\left(X\right)E\left(Y\right) = \dfrac{1}{5}$이다.

  \item [(3)] $\mathrm{Corr}\left(X, Y\right) = \dfrac{\mathrm{Cov}\left(X, Y\right)}{\sigma_X \sigma_Y} = \dfrac{\sqrt{1995}}{114} \approx 0.3918$
\end{itemize}

% Problem 5
\subsubsection{5.} $E\left(X\right)=3$, $E\left(Y\right)=2$, $E\left(X^2\right)=13$, $E\left(Y^2\right) = 7$ 그리고 $E\left(XY\right)=3$이다.

\begin{itemize}
  \item [(1)] 공분산 $\mathrm{Cov}\left(X, Y\right)$를 구하라.
  \item [(2)] 공분산 $\mathrm{Cov}\left(X-Y, X+Y\right)$를 구하라.
  \item [(3)] $X$와 $Y$의 상관계수를 구하라.
\end{itemize}

\paragraph{Solution.} \begin{itemize}
  \item [(1)] $\mathrm{Cov}\left(X, Y\right) = E\left(XY\right) - E\left(X\right)E\left(Y\right) = -3$\\
  \item [(2)]
\begin{align*}
	& \mathrm{Cov}\left(X-Y, X+Y\right) \\
	=& E\left(\left(X-Y\right)\left(X+Y\right)\right) - E\left(X-Y\right)E\left(X+Y\right) \\
	=& E\left(X^2-Y^2\right) - \left[E\left(X\right)-E\left(Y\right)\right]\left[E\left(X\right)+E\left(Y\right)\right] \\
	=& E\left(X^2\right) - E\left(Y^2\right) - \left[\left(E\left(X\right)\right)^2\right] + \left[\left(E\left(Y\right)\right)^2\right]\\
	=& 13-7-9+4=1
\end{align*}
  \item [(3)] $\sigma_X^2 = E\left(X^2\right) - \left(E\left(X\right)\right)^2 = 4$, $\sigma_Y^2 = E\left(Y^2\right) - \left(E\left(Y\right)\right)^2 = 3$이므로 $\mathrm{Corr}\left(X, Y\right) = \dfrac{\mathrm{Cov}\left(X, Y\right)}{\sigma_X \sigma_Y} = -\dfrac{3}{2\sqrt{3}} = -\dfrac{\sqrt{3}}{2}$이다.
\end{itemize}

% Problem 7
\subsubsection{7.} 두 확률변수 $X$와 $Y$의 결합확률밀도함수가 다음과 같다.

\[f\left(x, y\right) = \left\{
\begin{array}{ll}
	x+y & \qquad 0<x<1, 0<y<1 \\
	0 & \qquad\textrm{다른 곳에서}
\end{array}
\right. \]

\begin{itemize}
  \item [(1)] $X$와 $Y$의 공분산을 구하라.
  \item [(2)] $X$와 $Y$의 상관계수를 구하고, 상관관계를 확인하라.
  \item [(3)] 기댓값 $E\left(X-2Y\right)$와 분산 $Var\left(X-2Y\right)$를 구하라.
\end{itemize}

\paragraph{Solution.} $X$의 평균:
\begin{align*}
	& E\left(X\right) \\
	=& \int_0^1 \int_0^1 xf\left(x, y\right) \mathop{dy} \mathop{dx} \\
	=& \int_0^1 \int_0^1 x\left(x + y\right) \mathop{dy} \mathop{dx} \\
	=& \int_0^1 x\left(x + \dfrac{1}{2}\right) \mathop{dx} \\
	=& \dfrac{1}{3} + \dfrac{1}{4} = \dfrac{7}{12}
\end{align*}

$X^2$의 평균:
\begin{align*}
	& E\left(X^2\right) \\
	=& \int_0^1 \int_0^1 x^2f\left(x, y\right) \mathop{dy} \mathop{dx} \\
	=& \int_0^1 \int_0^1 x^2\left(x + y\right) \mathop{dy} \mathop{dx} \\
	=& \int_0^1 x\left(x^2 + \dfrac{1}{2}\right) \mathop{dx} \\
	=& \dfrac{1}{4} + \dfrac{1}{6} = \dfrac{5}{12}
\end{align*}

$X$의 분산은 $\dfrac{11}{144}$이다. 같은 방법으로 $Y$의 평균은 $\dfrac{7}{12}$, $Y$의 분산은 $\dfrac{11}{144}$이다.
\begin{itemize}
  \item [(1)] $XY$의 평균:
\begin{align*}
	& E\left(X\right) \\
	=& \int_0^1 \int_0^1 xyf\left(x, y\right) \mathop{dy} \mathop{dx} \\
	=& \int_0^1 \int_0^1 xy\left(x + y\right) \mathop{dy} \mathop{dx} \\
	=& \int_0^1 x\left(\dfrac{1}{2}x + \dfrac{1}{3}\right) \mathop{dx} \\
	=& \dfrac{1}{6} + \dfrac{1}{6} = \dfrac{1}{3}
\end{align*}

이므로, $\mathrm{Cov}\left(X, Y\right) = E\left(XY\right) - E\left(X\right)E\left(Y\right) = -\dfrac{1}{144}$이다.\\

  \item [(2)] $\mathrm{Corr}\left(X, Y\right) = \dfrac{\mathrm{Cov}\left(X, Y\right)}{\sqrt{Var\left(X\right)Var\left(Y\right)}} = -\dfrac{1}{11}$이다. $X$와 $Y$는 음의 상관관계를 가진다.\\
  \item [(3)] $E\left(X-2Y\right)	= E\left(X\right) - 2E\left(Y\right)	 = -\dfrac{7}{12}$, $Var\left(X-2Y\right) = Var\left(X\right) + 4Var\left(Y\right) - 4\mathrm{Cov}\left(X, Y\right) = \dfrac{59}{144}$
\end{itemize}